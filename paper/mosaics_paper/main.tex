\documentclass[a4paper,twocolumn]{revtex4}
%\documentclass[a4paper,preprint]{revtex4}
\usepackage{amsmath}
\usepackage{amsfonts}
\usepackage{amssymb}
\usepackage{graphicx}
\usepackage{subscript}
\usepackage{siunitx}
\usepackage{xcolor}
\usepackage{gensymb}


% * <carinanatalia.karner@gmail.com> 2018-08-17T16:06:11.135Z:
%
% ^.
\begin{document}

\title{Mosaics of patchy rhombi: from close-packed arrangements to open lattices}

\author{Carina Karner}
\email{carina.karner@univie.ac.at}
\affiliation{Faculty of Physics, University of of Vienna, Boltzmanngasse 5, A-1090, Vienna, Austria}

\author{Emanuela Bianchi}
\email{emanuela.bianchi@tuwien.ac.at}
\affiliation{Faculty of Physics, University of of Vienna, Boltzmanngasse 5, A-1090, Vienna, Austria}

\author{Christoph Dellago}
\affiliation{Faculty of Physics, University of of Vienna, Boltzmanngasse 5, A-1090, Vienna, Austria}

\date{\today}

\begin{abstract} 
Abstract here.
\end{abstract}

\maketitle

\section{Introduction}

Over the last decades, the availability of monodisperse non-spherical particles at the nano- and micro-scale~\cite{Sacanna2011,Vutukuri2014,Miriam2015} has allowed to design a vast variety of structures with different symmetries and packing densities~\cite{DeGraaf2012,Damasceno2012,Schultz2015}. 
%Complex spheroidal shapes range from colloidal molecules~\cite{Manoharan2003,Ravaine2012,Kraft2016} to more complex colloidal aggregates, such as multipod-like clusters of spheres~\cite{Ravaine2016}, while even more anisotropic particles range from convex units, such as rods~\cite{Kuijk2012}, cubes~\cite{Rossi2011} and polyhedral particles~\cite{Vutukuri2014}, to concave shapes, such as tetrapods and octapods~\cite{Miszta2011,Arciniegas2014} or bowl-shaped colloids~\cite{Marechal2010}.
The particle shape plays an important role in the self-assembly of target structures with tailored properties. 
%A systematic study on the assembly behavior of polyhedral hard particles of many different shapes has, for instance, allowed to group polyhedra into four categories of organization: liquid crystals, plastic crystals, crystals, and disordered (glassy) phases~\cite{Damasceno2012}. The same systems can be also mapped according to the coordination number in the fluid phase (\textit{i.e.}, the number of nearest neighbors surrounding each polyhedron in the fluid) and the shape factor (which measures the deviation from a sphere)~\cite{DeGraaf2012}, thus providing a roadmap to drive the assembly into the desired direction. 
%At the other side of the spectrum, colloidal branched nanocrystals, like tetrapods or octapods, tend to self-align on a substrate due to their geometry, making them interesting units for technological applications both in two- and three-dimensions~\cite{Miszta2011,Arciniegas2014}. Finally, the particle shape can also be engineered in order to promote lock and key interactions~\cite{Sacanna2013natmat,Wang2014,Ravaine2015}: shapes can be exploited to drive selective lock-and-key binding between colloids, thus realizing a simple mechanism of particle recognition and bonding.

Beyond the already rich framework offered by non-spherical particles, the interplay between the anisotropy of the building blocks and well-defined bonding sites on the particle surface might open tantalizing new perspectives. 

At the quasi two-dimensional level, the combination of shape- and bond-anisotropy has recently proven to direct the emergence of a rich assembly scenario.  A combined numerical and experimental investigation has shown that unconventional long-range ordered assemblies can be obtained for a class of highly faceted planar nanocrystals only when combining the effect of shape anisotropy with directional bonding~\cite{Ye2013}. Numerical investigations have also shown that regular polygonal nanoplates can be designed to assemble into many different Archimedean tilings~\cite{Millan2014}: the competition between shape anisotropy and interaction patchiness was tuned to obtain either close-packed or open tilings, the latter emerging mainly in binary mixtures of different shapes. Moreover, a systematic numerical study of the assembly of convex hard nanoplates combined attractive edge-to-edge interactions with shape transformations~\cite{Millan2015}: results were divided into space filling (not necessarily regular) tilings, porous (periodic) tilings and complex (disordered) tilings, thus providing an important insight into how shape and attractive interactions can be exploited to target specific tiling architectures. The development of heuristic rules for the design of two-dimensional superlattices would allow experimentalists to improve the crystal properties of already available structures as well as to access exotic phases with new interesting features.   Progress in the synthesis of anisotropic nanoplates offers indeed many possibilities for the rational design of two-dimensional materials with, for instance, different optical and catalytic properties~\cite{Efros2011,Gratzel2011}.  It is worth noting that hard polyhedral tiles decorated by attractive patches can be also used to describe two-dimensional molecular networks~\cite{Blunt2008}: numerical investigations on rhombus tilings have for instance identified the mechanisms leading to the emergence of ordered or random phases~\cite{Whitelam2012}. This study was successively extended by considering model molecules with particular rotational symmetries and studying their self-assembly into network structures equivalent to rhombus tilings~\cite{Whitelam2015}. Along similar lines, the assembly under planar confinement of patchy rhombi with a fixed geometry -- inspired by recently  synthesized particles~\cite{Miriam2015} -- has been investigated~\cite{Carina}, focusing on how the number, the type and the position of the patches along the edges of the rhombi influence the tilings (see Figure~\ref{fig:patchypolyhedra}).

%In the bulk, the combination of shape- and patch-induced  directional binding has recently provided amazing examples of tunable ordered structures.  Binary mixtures of different shapes with mutual attraction induced by isotropically distributed, complementary DNA strands can already exhibit a wide range of exotic extended architectures~\cite{Lu2015,diamond_origami} (see also section~\ref{sec:DNA}): (i) anisotropic polyhedral blocks and spheres, for instance, assemble into complex superlattices that can be tuned by the choice of the DNA shells and the particle size mismatch between the two components of the mixtures~\cite{Lu2015}; (ii) rigid tetrahedral DNA origami cages and spherical nano-particles can form a family of lattices based on the diamond motif~\cite{diamond_origami}. When focusing on one-component systems of anisotropic particles decorated with anisotropic bonding patterns, most of the results accumulated so far in the literature deal with Janus-like non-spherical entities -- mainly elongated shapes carrying one or at most two patches -- assembling into a vast variety of fiber-like structures with diverse applications. Janus nano-cylinders that form vertical, horizontal or even smectic arrays~\cite{Smith2013,Smith2014}, ellipsoids with one patch in a Janus-like or ``kayak" fashion that form ordered assemblies~\cite{Shah2013} or even field-sensitive colloidal fibers~\cite{Shah2014},``Mickey Mouse''-shaped colloidal molecules that form tubular aggregates~\cite{Kraft2015}, and silica rods coated with gold tips that self-assemble into different multipods~\cite{Chaudhary2012} are just few examples. The susceptibility of Janus-like anisotropic units to external fields can also be used to drive the assembly into string-like structures~\cite{ShieldsIV2013}.

%Finally, it is worth noting that the combined approach of complex shapes and bonding surface patches can provide insights into biological processes~\cite{DenOtter2010}. 


\section{System}
\subsection{Particles}
In this work we explore the phases of hard $60\degree$ rhombi in two dimensions with four square-well patches and two different kinds of patches.
Patches of the same kind attract each other with a strength $-\epsilon$ and patches of a different kind repel each other 
with $\epsilon$ (4/2-2 systems).
The three ways to distribute four patches of two types on four rhombus edges are:
Positioning patches of the same type to enclose the larger rhombus angles (double-manta, dma), positioning patches of the same type to enclose the smaller rhombus angles ( double-mouse, dmo) and 
positioning patches of the same type to sit on the parallel edges (checkers).
In all of these systems the patches can still be placed anywhere on the edge.
To explore this - in principle - infinite parameter space systematically we define movement patterns or so-called patch topologies. 
They describe the relationship of the patch positions ($\Delta_{1}, \Delta{2}, \Delta{3}, \Delta{4}$) to each other. 
Note that we define patch positions as distances to a distinguished vertex.
For example, the s1 topology of dma and dmo systems describes all systems where patches of the same kind have the same distance 
to the vertex enclosed by the patches.
We then investigate systems in this topology by varying $\{\Delta_{k}\}$.
In general, the investigated topologies can be split into symmetric topologies, where patches of the same type $m$ retain the same $\Delta$ ($\Delta_{m2} = \Delta_{m1}$) and asymmetric topologies where $\Delta_{m2} = 1 - \Delta_{m1}$.
With these definitions a particle can be characterized by its patch topology and only one $\Delta$.
Note that all topologies collapse to their respective central topology for $\Delta = 0.5$ ( dma-c, dmo-c, checkers-c). 

%--------- FIG 1 -------------------------------------------

\begin{figure*}
\begin{center} 
\includegraphics[width=\textwidth]{figures/phases.png}
\caption{}5
\label{fig:phases}
\end{center} 
\end{figure*}

%-----------------------------------------------------------

\subsection{Simulation details }
To model the absorption of 2D colloids on a surface we chose the grand canonical ensemble ($\mu VT$) with single particle rotation and translation moves and particle insertion and deletion. 
To avoid kinetic traps we implemented cluster moves \cite{XX}.
To start the self assembly process we followed the same procedure in all investigated systems:
After equillibration of $\approx 10^5$ MC-sweeps in a regime of very low concentrations of $\phi\approx 0.05$ with $\mu_{eq}$ we increase the chemical potential to $\mu^{*}$ to observe nucleation processes. 
Typical system sizes of the completed assembly are in the range of $N\approx 1000$.
Numerical values of the system parameters are given in the supporting information.
% how many runs per system. 

\section{Results}
%TALK about reversing logic/chirality (same by patch type exchange, same by bla (asym2 0.2,0.8 are chiral!) 

The starting point for our explorations is the system investigated by Whitelam et al.\cite{Whitelam} and we implemented it as reference.
In their work the system of $60\degree$ rhombi with four equal patches placed in the center(4-eq-c) assembles into a random tiling and successfully models the assembly of TPTC and similar small organic molecules.
The reason why this simple model predicts the phase of TPTC so well is because just like patchy $60\degree$ rhombi, TPTC molecules can bind to one another in three possible orientations, one of them parallel and two non-parallel. 
The random tiling results from the fact that in the 4-eq-c system particles are equally likely to bind parallel or non-parallel. 
%More abstractly speaking, random tilings are tilings without positional order that show algebraic correlations of bonded particle orientations as function of distance.
But as there are binding restrictions put on particles attaching to already bound dimers/multimers, the distribution of parallel and non-parallel bonds 
is not equal. To account for this imbalance and to quantify the randomness correctly, the order parameter 
$\Psi$ is used with $\Psi = (0.608 n_{p} - 0.392 n_{np}) / (0.608 n_{p} + 0.392 n_{np})$, 
where $n_{p}$ is the total number of parallel bonds in the largest cluster
and $n_{np}$ the number of non-parallel bonds. The numerical values were estimated from simulation in \cite{Stannard2012} (see supporting information).

Analogous to Whitelam et al. (w.) we obtain $\Psi\approx 0$ for the 4-eq-c system, where $\langle\Psi_{w.}\rangle = -0.08 \pm 0.05$ and $\langle\Psi_{\text{comparison}}\rangle= 0.02\pm0.04]$ (see Fig. 
\ref{fig:domain_sizes}).
% Discuss reason for discrepancy

%--------- FIG 2 -------------------------------------------

\begin{figure*}
\begin{center} 
\includegraphics[width=\textwidth]{figures/phase_diagram.png}
\caption{}5
\label{fig:phase_diagram}
\end{center} 
\end{figure*}

%-----------------------------------------------------------

Moving past this reference point, in the following we discuss the phases of the already defined 4/2-2 systems, with different symmetric and asymmetric patch topologies (see Fig.\ref{fig:phases}).
We calculated phase diagrams for all defined systems (see Fig.\ref{fig:phase_diagram}).
Each phase diagram consists of phase points observed at different interaction strengths $\epsilon$ and patch positions $\Delta$.
We characterized the emerging phases with the randomness parameter $\Psi$. 
% Note what phase diagram means here. 
% how was av psi calculated for phases and non-phases. 
% mention phases versus non-phases.

In all systems and corresponding to \cite{Whitlam}, the sweet spot for phase formation at the selected state $\mu^{*}VT$ is at $\epsilon = 5.2 k_{B}T$. For interaction strengths $\epsilon \leq  4.2 k_{B}T$ all investigated systems remain liquid, while for $\epsilon \geq 6.2$ we observe a tendency for defects due to low energy barriers and percolating networks or random cluster phases due to high interaction energies.

Another common aspect is that in dma and dmo systems the phases resulting from central patch topologies are different from their respective off-center topologies. This can observed qualitatively in the abrupt color change in the phase diagrams in Fig.\ref{fig:phase_diagram}.

The most interesting general pattern we found is that symmetric patch topologies lead to closed lattices, whereas asymmetric patch topologies lead to parallel and non-parallel open lattices with pore sizes dependent on the patch positions ( see Fig.\ref{fig:phases} and Fig.\ref{fig:domain_sizes}).
We find open systems in all three system types apart from the patch specificity it is always the same patch topology (as2) that leads 
to the open lattices. 

With these general phase behaviour patterns revealed, we proceed with discussing how these arise from the underlying particle topologies.
To understand the rationale behind phase formation we rely heavily on Fig. ref{fig:phases}, whereas the phase characteristics are mainly displayed in Fig.\ref{fig:domain_sizes} and Fig.\ref{fig:phase_diagram}. 
If not otherwise stated we discuss phase points at $\epsilon=5.2 k_{B}T$.

\textbf{Phases of the center topology.}
We studied dma-c, dmo-c and checkers-c, where all patches are placed in the edge center.
All three systems yield closed, space tiling lattices but only the dma-c yields a random tiling with $\langle\Psi\rangle\approx 0$.
In the dmo-c phase on the other hand, parallel bonds dominate with $\langle\Psi\rangle=0.52\pm 0.02$, and the checkers-c yields a defect free parallel lattice with $\langle\Psi\rangle = 1.00$. 
Dma-c tiles randomly because just as 4-eq-c, the double manta specificity allows for multi-particle parallel, non-parallel as well mixed bonded clusters (3-np, 4-p and 7-p$\&$np in Fig.\ref{fig:phases}b).

In dmo-c, although parallel as well as non parallel pair bonds are allowed, 3-np configurations induce an energetic penalty. This restricts the phase to either parallel clusters, (e.g. 4-p), roof-shingles (e.g 8-p$\&$np) or mixtures of those. 
Note however, that the close-off of roof-shingle motives 
inevitably induces either energetic penalties (9-p$\&$np) or defects (see Fig.\ref{fig:phases}a whenever three orientations coalesce). 

The checkers-c system yields a parallel phase because only parallel bonds are possible without energetic penalities (see 2-p 
versus 2-np). 
This holds true for all topologies of the checkers systems and 
for most phase points, this restriction leads to defect free parallel lattices. 

% Equivalence of 0.2 and 0.8 etc
% Equivalence by rotation, chirality, etc.

%--------- FIG 3 -------------------------------------------

\begin{figure*}
\begin{center} 
\includegraphics[width=1.0\textwidth]{figures/pore_psi_domain.png}
\caption{}5
\label{fig:domain_sizes}
\end{center} 
\end{figure*}

%-----------------------------------------------------------

\textbf{Phases of symmetric off-center topologies.}
Symmetric off-center topologies are those where patches of the same kind either have the same distance to their enclosing angle (dma-s1/s2, dmo-s1/s2) or remain parallel (checkers-s1). 
For the dma and dmo systems off-center topologies result in very different phases to the center patch topology:
Moving patches symmetrically off-center in those systems results in parallel pair bonds with offset edge contact (see Fig.\ref{fig:phases}b). This forbids the formation of parallel multi-particle clusters in both classes of systems. 

In the dma-s1 and dma-s2 the formation of 3-np is still possible and this results in defect-free non parallel phases with $\langle\Psi\rangle=-1.00$.
In the off-center dmo-s1/2, equivalently to dmo-c, the formation of 3-np is forbidden energetically and since the formation roof-shingles is is not possible due to the offset edge contact, dmo-s1/s2 do not form any coherent phase.

In checkers-s1, besides a full-edge contact parallel phase (e.g 4-p), off-center patch positioning additionally allows for an offset parallel tiling brick configuration (e.g 4-brick). Subsequently, the checkers-s1 assemblies turn out to be parallel tilings with occasional brick defects ($\langle\Psi\rangle = 1.00$).

%For symmetric off-center double mouse systems we observe disconnected clusters of 6-particle “star” clusters and %parallel chains. Neither of these  building blocks can connect to form a tiling phase.

% domain sizes:
% in random tiling: np domains bigger and a fraction of np domains bigger
% in dmo-tiling: np domains small (2,3,4) but fraction of np domains high (many roof shingle 2np domains), paralllel domains bigger
% exponential distribution of domain sizes 

\textbf{Phases of asymmetric off-center topologies.}
Asymmetric off-center topologies result either from moving patches of the same type asymmetrical with respect to their enclosing angle (dma-as1/2 and dmo-as1/2) or such that they are not parallel when sitting on opposing edges (checkers-as1/as2). 
We find that the as2 topology, where all patches are placed asymmetrical with respect to all vertices, yields open lattices with pore sizes dependent on $\Delta$ for all three system types. 
The edge contact is now offset for parallel and non-parallel bonds in all systems. 

In the dma-as2 systems, 3-np clusters with holes form and these connect with other 3-np clusters to a hexagonal 3-np-triangular-pore lattice. The resulting lattice has hexagonal and triangular pores with their size dependent on $\Delta$ such that the pores are biggest for $\Delta$ close to 0 or 1 ( see Fig.\ref{fig:domain_sizes}). At $\Delta=0.5$ the patch topology collapses back to the center topology (dma-c) with a closed random tiling as phase.

In the dmo-as2 and the checkers-as2, parallel offset bonding pairs connect to an open parallel lattice with rhombic pores (open rhombi lattice).
In those two system families a non-parallel lattice is impossible. In the dmo-as2 3-np clusters yield unsatisfied unsatisfied bonds and the checkers-as2 can not even form non-parallel pair bonds.

In the sticky limit the side length $pl$ for pores in both open lattices is given by 
\begin{equation}
pl = l - 2\Delta,
\end{equation}
where $l$ is the side length of the rhombus. 
The respective pore areas are then given by the areas of the rhombi,  the triangle and the hexagon with side length $pl$. The dependence of the pore areas on $\Delta$ in the sticky limit is given in Fig.\ref{fig:domain_sizes}.
% Explain why the cusp in the triangle. Discuss sticky packing fraction
% and actual packing fraction.

In the dma-as2 a 4-p-rhombic-pore lattice is energetically and configurationally possible and is in fact taken on in 
1/16 simulation runs with $\Delta=0.2$.
An open phase with mixed parallel and non-parallel bonds is never taken on for $\Delta\in[0.2,0.8,0.3,0.7]$ as the connection of  3-np-triangular-pores with 4p-rhombi-pores induces a high strain on lattices with pore sizes that big.
This effect leads to self-healing, i.e. whenever a 4-p-rhombic-pore forms during crystallization, it quickly self-heals to a 3-np-pore rather quickly.

For smaller pore sizes, at $\Delta\in[0.4,0.6]$ we do observe 4-p-rhombi-pore defects within the 3-np-triangular-pore lattice, an observation that is reflected in the finite domain size of non-parallel domains in dma-as-0.6 and in $\langle\Psi\rangle=-0.76\pm0.18$ (see Fig.\ref{fig:domain_sizes}).

Dma-as1 and dmo-as1 are defined through placing patches of the same kind asymmetrical with respect to the enclosing angle, but patches of different kind symmetrical. 
For both system families the as1-topology does not result in a coherent phase. 
In the dma-as1 3-np clusters can be formed with all bonds satisfied, but those are unable to connect with each other to due to overlaps.  
In the case of dmo-as1, neither 3-np nor 4-p clusters can form without overlaps or unsatisfied bonds. 

The checkers-as1 topology is defined by placing patches of onetype asymmetrical, while patches of the other type remain in the center. This topology yields a brick-phase (see Fig.\ref{fig:phases}b, 4-brick).  

%Apart from single particle attachment in the nucleation of the open-star phase, we also observe hierarchal %crystallisation: 
%Open three-particle boxes form and attach to growing open-star clusters (see video in supplementary info and %Footnote 2).
%The more extreme the $/Delta$ value the faster this phase transition takes place. (Theory why?).
%In the systems that lead to closed packed lattices and the open parallel lattice we found nucleation to proceed solely via single particle attachment. is this true tho? 

\textbf{Defects and high energy regime.}
%measures: cluster sizes, snapshots
%dma-s1 and dma-s2, defect free
%double manta more prone to defects than double mouse, double mouse more prone to defects than checkers at %higher energies.
%Double manta has parallel defects off-center. In fact parallel phase is also possible and sometimes taken on. 
%closed double manta not likely for defects (parallel bonding not possible) 
%the non-phases tend to more balanced out parallel non parallel binding.
%Reason for small energies clusters are not big, pair bonds parallel and non-parallel equally likely. 
%At higher packing fraction non parallel becomse more likely 
%in double mouse Open five/6 star networks (non parallel)

\footnotesize{
\begin{thebibliography}{9}
\bibliographystyle{rsc} 
\bibitem{Whitelam2012}
 S. Whitelam, I. Tamblyn,  P.H. Beton, J.P. Garrahan,
\emph{Random and Ordered Phases of Off-Lattice Rhombus Tiles},
Physical Review Letters, Vol. 108/3,
2012 
  
\bibitem{Elemans2009}
A. Elemans, S. Lei, S. De Freyter,
\emph{Molecular and Supramolecular Networks on Surfaces: From Two-Dimensional Crystal Engineering to Reactivity},
Angewandte Chemie International Edition Vol. 48/40,
2009

  
\bibitem{Whitelam2015}
 S. Whitelam, I. Tamblyn, J.P. Garrahan, P.H. Beton,
\emph{Emergent rhombus tilings from molecular interactions with M-fold rotational,
symmetry},
Physical Review Letters, Vol. 114/11,
2015  

\bibitem{Bartels2010}
L. Bartels,
\emph{Tailoring molecular layers at metal surfaces},
Nature Chem. Vol. 2/87,
2010

\bibitem{Blunt2008}
M. Blunt et al.,
\emph{Random Tiling and Topological Defects in a Two-Dimensional Molecular Network},
Science Vol. 322/1077,
2008

\bibitem{Whitelam2007}
S. Whitelam, P.L. Geissler,
\emph{Avoiding  unphysical kinetic traps in Monte Carlo simulations of strongly attractive particles},
J. Chem. Phys.Vol 127,
2007

\bibitem{Whitelam2010}
S. Whitelam,
\emph{Approximating the dynamical evolution of systems of strongly interacting overdamped particles},
Molecular Simulation Vol. 37/7,
2011

\bibitem{Stannard2012}
A. Stanndard,
\emph{Broken symmetry and the variation of critical properties in the phase behaviour of supramolecular rhombus tilings},
Nature Chem. Vol 4/112,
2012

\bibitem{Jockusch1998}
W. Jockusch, J. Propp, P. Shor,
\emph{Random Domino Tilings and the Arctic Circle Theorem},
arXiv:math/9801068,
1998

\end{thebibliography}

%bibliography{patchy-review-uniq} 
}

\end{document}

